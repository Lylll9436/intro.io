\documentclass[10pt, letterpaper]{article}

% ======= Packages and settings =======
\usepackage[utf8]{inputenc}
\usepackage{CJKutf8}
\usepackage[
    ignoreheadfoot,
    top=2.5cm, bottom=2.5cm,
    left=2cm, right=2cm,
    footskip=1.0cm
]{geometry}
\usepackage{titlesec}
\titleformat{\section}
  {\needspace{4\baselineskip}\bfseries\LARGE}
  {}{0pt}{}%
  [\vspace{-4pt}\titlerule]   % 这一行会画横线

\titlespacing{\section}{-2pt}{0.3cm}{0.2cm}
\usepackage{tabularx}
\usepackage{array}
\usepackage{setspace}
\setstretch{1.05}  % 全局行距调整
\usepackage[dvipsnames]{xcolor}
\definecolor{primaryColor}{RGB}{0, 79, 144}
\usepackage{enumitem}
\usepackage{fontawesome5}
\usepackage{amsmath}
\usepackage{times} % 使用Times New Roman字体
\usepackage[T1]{fontenc} % 确保字体编码正确
\usepackage[
    pdftitle={Yunlong_Liu - CV},
    pdfauthor={Yunlong_Liu},
    colorlinks=true,
    urlcolor=primaryColor
]{hyperref}
\pdfstringdefDisableCommands{%
  \def\faEnvelope[#1]{}%
  \def\faMapMarker*{}%
  \def\faPhone*{}%
  \def\faGithub{}%
  \def\faLinkedinIn{}%
  \def\textsubscript#1{#1}% 如果你用过 \textsubscript
}
\usepackage{paracol}
\usepackage{changepage}
\usepackage{lastpage}
\usepackage{needspace}

% ======= Style adjustments =======
\pagestyle{empty}
\setcounter{secnumdepth}{0}
\setlength{\parindent}{0pt}
\renewcommand\labelitemi{$\bullet$}
\newenvironment{highlights}{
    \begin{itemize}[
        topsep=0.01cm, parsep=0.01cm, partopsep=0pt, itemsep=0pt,
        leftmargin=2.1em, labelwidth=2em, labelsep=0em,
        align=left
    ]
}{\end{itemize}}

\newenvironment{onecolentry}{\begin{adjustwidth}{0.2cm}{0.2cm}}{\end{adjustwidth}}
\newenvironment{twocolentry}[2][]{%
    \onecolentry
    \def\secondColumn{#2}
    \setcolumnwidth{\fill, 4.5cm}
    \begin{paracol}{2}
}{%
    \switchcolumn \raggedleft \secondColumn
    \end{paracol}
    \endonecolentry
}

% ======= Document =======
\begin{document}
\begin{CJK*}{UTF8}{gbsn}

% ======= Header =======
\begin{center}
    {\fontsize{24pt}{24pt}\selectfont \textbf{Yunlong Liu}}\\[4pt]
    {\footnotesize Nanjing, Jiangsu, China $|$ \href{mailto:lyl_arch@seu.edu.cn}{lyl\_arch@seu.edu.cn}}
\end{center}

\vspace{-0.3cm}
\begin{center}
    \textit{Research Interests: Urban Analytics, Digital Twin, Graph Neural Network, Multimodal Fusion, Large Language Model}
\end{center}

% ======= Education =======
\section{Education}
\begin{onecolentry}
    \textbf{Southeast University}, Nanjing, China \hfill Sep 2023 -- Jun 2026\\
    M.Arch in Urban Design, GPA: 3.7/4.0\\
    \textit{Relevant Coursework:} Programming Fundamentals for Architecture (90), Big Data Methods and Applications in Urban Planning (92), Urban Big Data Analysis and Intelligent Applications (91), Digital Technology Applications in Historic Environments (90)
\end{onecolentry}

\vspace{0.3\baselineskip}
\begin{onecolentry}
    \textbf{Shandong Jianzhu University}, Jinan, China \hfill Sep 2018 -- Jun 2023\\
    B.Arch in Architecture, GPA: 3.3/4.0\\
    \textit{Relevant Coursework:} Advanced Mathematics (91), Computer Fundamentals (92)
\end{onecolentry}

% ======= Research Experience =======
\section{Research Experience}

% --- National Key R&D Program ---
\begin{onecolentry}
\textbf{National Key R\&D Program of China — Automatic generation and optimization of city-scale green performance models} \hfill Nov 2024 -- Present\\
\textit{Graduate Research Assistant}
  \begin{highlights}
    \item \textbf{My Role:} Designing system architecture and integrating optimal models for multimodal data integration\\[-12pt]
    \item \textbf{Method:} Building LLM-enhanced urban knowledge graph to structure heterogeneous data, then training GNN on this graph to extract node features for urban performance prediction and optimization\\[-12pt]
    \item \textbf{Progress:} Completed initial framework design and data preprocessing pipeline, currently optimizing model performance at city scale
  \end{highlights}
\end{onecolentry}

\vspace{0.15cm}

% --- Independent Study ---
\begin{onecolentry}
\textbf{Independent Study: SAM-GPT semantic enhancement framework for street view imagery analysis} \hfill Nov 2024\\
\textit{Self-directed Research}
  \begin{highlights}
    \item \textbf{Contribution:} Developed framework addressing LLM inaccuracy in object localization for urban analytics\\[-12pt]
    \item \textbf{Method:} Integrated SAM's panoptic segmentation masks as spatial guidance for GPT models to improve object recognition accuracy\\[-12pt]
    \item \textbf{Achievement:} Achieved significant improvement in LLM object description accuracy; framework adopted by research colleagues for Architectural Society of China 2025 Annual Conference presentation
  \end{highlights}
\end{onecolentry}

\vspace{0.15cm}

% --- Master's Thesis Research ---
\begin{onecolentry}
\textbf{Master's Thesis Research: Multimodal data fusion for urban carbon emission simulation model — A case study of Nanjing} \hfill Apr 2024 -- Present\\
\textit{Independent Research under Prof. Yu Zhang}
  \begin{highlights}
    \item \textbf{Contribution:} Designing multimodal fusion framework for city-scale carbon emission prediction, focusing on Nanjing as case study\\[-12pt]
    \item \textbf{Method:} Developing multimodal fusion strategy integrating meteorological time series, remote sensing, and statistical data with attention mechanism optimization\\[-12pt]
    \item \textbf{Progress:} Completed data collection from four modalities and framework design; implementing attention mechanism and model training
  \end{highlights}
\end{onecolentry}

% ======= Publications =======
\section{Publications}

% --- Publication 1: Decoding the 24-hour city ---
\begin{onecolentry}
\textbf{Y. Liu} and Y. Zhang, ``Decoding the 24-hour city: A framework for cross-domain impact prediction integrating knowledge graph and graph neural network,'' accepted for oral presentation at \textit{the 59th ISOCARP World Planning Congress}, Riyadh, Saudi Arabia, Dec. 2025. \textbf{[Oral presentation forthcoming]}
  \begin{highlights}
    \item \textbf{Contribution:} Developed novel framework combining knowledge graphs and GNNs to integrate multimodal urban data (remote sensing, GIS, energy consumption) for predicting impacts across three urban sectors\\[-12pt]
    \item \textbf{Method:} Built urban ontology for data integration, constructed semantically-enriched knowledge graph, and implemented semantic-aware GNN for cross-domain reasoning\\[-12pt]
    \item \textbf{Achievement:} Achieved about 20\% improvement in cross-domain prediction accuracy, enabling planners to anticipate cascading policy effects across urban systems
  \end{highlights}
\end{onecolentry}

\vspace{0.15cm}

% --- Publication 2: From pixels to predicates ---
\begin{onecolentry}
\textbf{Y. Liu}, S. Li, P. Liu, Y. Zhang, and R. Stouffs, ``From pixels to predicates: Structuring urban perception with scene graphs,'' abstract accepted at \textit{31st International Conference of CAADRIA}, 2026. \textbf{[Abstract accepted; Full paper submitted]}
  \begin{highlights}
    \item \textbf{Contribution:} Developed scene graph-based framework transforming street view imagery into structured relational data, bridging gap between low-level pixel analysis and high-level semantic understanding in urban analytics\\[-12pt]
    \item \textbf{Method:} Applied PSGTR model to extract scene graphs from SVIs, then trained Heterogeneous Graph Autoencoder (HeteroGAE) with unsupervised pre-training on the Place Pulse dataset (1.2M images) to learn relational embeddings; validated generalization on New York and Chicago street view datasets\\[-12pt]
    \item \textbf{Achievement:} Identified novel micro-scale urban patterns through relational analysis, discovering safety-perception correlations like ``(graffiti)-[painted on]->(wall)'' that are undetectable by conventional pixel-based methods; demonstrated robust cross-city transferability
  \end{highlights}
\end{onecolentry}

\vspace{0.15cm}

% --- Publication 3: Chinese Superblocks Regeneration ---
\begin{onecolentry}
Y. Tang, \textbf{Y. Liu}, P. Wang, J. Yin, S. Li, P. Liu, L. Li, P. F. Yuan, and R. Stouffs, ``Chinese superblocks regeneration: Street network learning with an edge-centered unsupervised graph neural network,'' abstract accepted at \textit{31st International Conference of CAADRIA}, 2026. \textbf{[Abstract accepted]}
  \begin{highlights}
    \item \textbf{My Role:} Developed edge-centered unsupervised GNN framework analyzing 7,000 street segments across seven Chinese cities for superblock regeneration\\[-12pt]
    \item \textbf{Method:} Constructed ego-graphs with 250m buffer zones using route structure theory, employed multi-layer GCNs with contrastive learning to generate 128-dimensional embeddings\\[-12pt]
    \item \textbf{Achievement:} Identified four distinct structural types (accessibility hubs, transitional connectors, inward-facing branches, peripheral buffers) enabling targeted regeneration strategies and cross-city transfer
  \end{highlights}
\end{onecolentry}

\vspace{0.15cm}

% --- Publication 4: Regional Urban Development ---
\begin{onecolentry}
\textbf{Y. Liu} and Y. Zhang, ``Research on regional urban development level based on club convergence phenomenon: A case study of the Yangtze River Delta urban agglomeration,'' in \textit{Proceedings of the China Urban Planning Annual Conference}, June 2024. \textbf{[Published]}
  \begin{highlights}
    \item \textbf{Contribution:} Analyzed regional urban development patterns in Yangtze River Delta using club convergence theory\\[-12pt]
    \item \textbf{Method:} Applied Markov transition matrix and spatial Markov chain analysis to examine urban development trajectories and spatial spillover effects\\[-12pt]
    \item \textbf{Achievement:} Identified ``poverty trap'' risks in western cities while eastern cities showed higher upward mobility, providing strategic insights for regional policy
  \end{highlights}
\end{onecolentry}

% ======= Skills =======
\section{Technical Skills}
\begin{onecolentry}
    Programming \& ML: Python, Git, Graph Neural Networks, Deep Learning \\
    Spatial Analysis \& Data Processing: GIS (QGIS, ArcGIS), Remote Sensing, Multimodal Data Fusion, Knowledge Graph \\
    Languages: Mandarin Chinese (native), English (fluent)
\end{onecolentry}

% ======= Academic Activities =======
\section{Academic Activities \& Honors}
\begin{onecolentry}
    Participant, DigitalFutures Workshop – Habitat Representation: Graph Neural Networks for Urban Examination and Urban Design, Tongji University, 2025\\[0.05cm]
    Participant, DigitalFutures Workshop – Computational Design with shape Grammar, Tongji University, 2023\\[0.05cm]
    Participant, Southeast University–University of Pennsylvania Joint Summer Workshop, 2024\\[0.05cm]
    First Prize, University-level Regional Planning and Design Project — Shunde Village Redevelopment, 2024\\ [0.05cm]
    Academic Merit Scholarship, Southeast University, 2023, 2024\\[0.05cm]
    Outstanding Bachelor's Thesis Award, Shandong Jianzhu University, 2023
\end{onecolentry}

\end{CJK*}
\end{document}
